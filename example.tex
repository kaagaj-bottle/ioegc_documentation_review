%%%%%%%%%%%%%%%%%%%%%%%%%%%%%%%%%%%%%%%%%%%%%%%
% Working Paper Template Example for
% IOE Graduate Conference 
% Version 15.0
%
% Original author:
% Jayandra Raj Shrestha (jayandra@ioe.edu.np)
%%%%%%%%%%%%%%%%%%%%%%%%%%%%%%%%%%%%%%%%%%%%%%%

% use the option blindreview to mask the author details for blind review submission
% remove the blinereview option to make the author information visible
\documentclass[fleqn, 10pt, twoside, blindreview]{IOEGC}

\hypersetup
{
    pdfauthor   = {author-1, author-2, author-3...}    % Edit with actual author names
}

\begin{document}
\maketitle
\thispagestyle{firstpage} 

\section{Introduction}              
Nepal, a land of unparalleled biodiversity and stunning landscapes, is home to a richar-
ray of wildlife species that inhabit its lush forests, soaring mountains, and vibrante-
cosystems. The diverse wildlife, including endangered and elusive species like the
Bengal tiger, one-horned rhinoceros, and the elusive snow leopard, is a testament to the
country’s commitment to conservation. However, with the increasing challenges posed
by habitat loss, poaching, and climate change, preserving Nepal’s wildlife heritage de-
mands innovative and technologically advanced approaches. This project setsits sights
on creating a groundbreaking solution to bolster wildlife conservation efforts in Nepal
through an integrated approach for wildlife recognition and surveillance usingaudio and
video sensors. The synergy of audio and video data provides a comprehensive under-
standing of animal behavior, movement patterns, and species composition within their
natural habitats. By fusing these two streams of data, we aim to build arobust system
that can identify and monitor wildlife with a higher level of accuracy andefficiency.



\section{What is \LaTeX?} \label{sec:whatis}
\LaTeX{} is a document preparation system for the \TeX{}
typesetting program. It offers programmable desktop publishing features and 
extensive facilities for automating most aspects of typesetting and desktop  
publishing, including numbering and cross-referencing,
tables and figures, page layout, bibliographies, and much more. 

\begin{itemize}[noitemsep]
	\item A family of programs designed to produce \\publication-quality typeset
		documents.
	\item Particularly good at working with mathematical symbols.
	\item WYSIWYM\footnote{What You See Is What You Mean} rather than 
		WYSIWYG\footnote{What You See Is What You Get}.
\end{itemize}

The history of LaTeX begins with a program called \TeX. In 1978, a computer 
scientist by the name of \textbf{Donald Knuth} grew frustrated with the mistakes
that his publishers made in typesetting his work. He decided to create a 
typesetting program that everyone could easily use to typeset documents, 
particularly those that include formulae, and made it freely available. 

Knuth's product is an immensely powerful program, but one that does focus
very much on small details. A mathematician and computer scientist by the
name of Leslie Lamport wrote a variant of \TeX\ called \LaTeX\ that focuses on
document structure rather than such details.


\section{Getting the \LaTeX\ Software }\label{sec:getting}
There are two major standard distributions of \LaTeX:
\begin{itemize}[noitemsep]
	\item TeXLive \\ \texttt{\small{https://www.tug.org/texlive/}}
	\item MikTeX \\ \texttt{\small{https://miktex.org/}}
\end{itemize}

These are freely downloadable from the internet. TeXLive works in all the major 
PC platforms like Windows, Unix, Linux, and Mac. Whereas, MikTeX is for Windows 
only. When you install these, you also get the TeXWorks editor as your frontend.
More than a dozen other frontend GUIs are available for \LaTeX. 
Some of these are:

\begin{itemize}[noitemsep]
	\item TeXMaker \\ \texttt{\small{http://www.xm1math.net/texmaker/}}
	\item TeXnic Center\\ \texttt{\small{http://www.texniccenter.org/}}
\end{itemize}


\section{Template Structure} \label{sec:struct}
This \LaTeX\ template resides on a folder with the following files/folder:
\begin{description}
	\item[example.tex]		The main \LaTeX\ source file of this document (example.pdf). 
		Working Example on using the template with some description.
	\item[article.tex] A minimal alternate of example.tex.	
	\item[paperinfo.tex] Contains the abstract and other author information.
	\item[pagenum.tex]		Contains the code for starting page number which 
							will be edited during final compilation.
	\item[IOEGC.cls] 	\LaTeX\ class file for managing the styles and 
		formats of the document. Prohibited to edit.
	\item[refs.bib] 	File for placing the bibliography data in BibTeX format.
	\item[Graphics] 		Folder for keeping all the final graphics files 
							(.jpg, .png, etc.) used in the document.
\end{description}


\section{Sections}
Paragraphs within a document can be separated just by leaving one blank line 
between them.

\LaTeX\ supports section headings upto 3 levels via the following commands:

\begin{itemize}[noitemsep]
	\item \verb+\section{...}+
	\item \verb+\subsection{...}+
	\item \verb+\subsubsection{...}+
\end{itemize} 

These have been illustrated properly in section \ref{sec:lists} of this example.
You can use their starred variants given below to suppress section numbering 
which has been demonstrated in the \emph{Acknowledgment} section.

\begin{itemize}[noitemsep]
	\item \verb+\section*{...}+
	\item \verb+\subsection*{...}+
	\item \verb+\subsubsection*{...}+
\end{itemize} 


\section{Typesetting Mathematics}
\LaTeX\ has very rich features for typesetting mathematics. Please refer to 
\LaTeX\ and AMSmath manuals or online resources for further information. 
Here are a few examples.

The formula given in equation \ref{eq:quad} can be used to determine the roots 
of a quadratic equation of the form: $$ax^2+bx+c=0$$
Here, $a$, $b$, and $c$ are constants/coefficients and $x$ is a variable.

Numbered equation:
\begin{equation}
	x= \frac{-b\pm \sqrt{ b^2-4ac}}{2a}
	\label{eq:quad}
\end{equation} 

Equation without a number
\begin{equation*}
	x = \frac{-b\pm \sqrt{ b^2-4ac}}{2a}
\end{equation*} 

\section{Creating Tables}
Table \ref{tbl:capacity} is an example of a simple table in \LaTeX. To create 
complex tables, please refer to \LaTeX\ manuals or online resources. Use 
\verb+\begin{table*}+ to take up the entire page width. However, the use of 
tables spanning the entire page width is discouraged as it needs extra caution.

\begin{table}[H]
	\caption{No. of papers presented in IOEGC}
	\label{tbl:capacity}
	\centering
	\begin{tabular}{|c|l|c|} %three columns: left, center, center aligned
		\hline
		SN & Year & No. of papers \\
		\hline
		1 & 2013 & 42	\\
		2 & 2014 & 79	\\
		3 & 2015 & 46	\\
		4 & 2016 & 49	\\
		5 & 2017 & 83 	\\
		\hline
		 & Total & 299 	\\
		\hline
	\end{tabular}
\end{table}


\section{Placing Figures} \label{sec:figures}
One can generate technical graphs or diagrams from \LaTeX\ also, but this 
requires another level of expertise. Another alternate is to use R-programming 
code to generate graphs on the fly thus producing reproducible documents, 
which requires S-Weave. However, it is very common to include figures generated 
from other sources or programs. Here are a few examples on placing figures with 
proper captioning and label for cross referencing. The most suitable format for 
figure files to produce final output in raster format as pdf are:

\begin{enumerate}[noitemsep]
	\item PNG
	\item PDF
	\item JPG
\end{enumerate}

\begin{figure}[H]\centering
	% Remove fbox if you donot require a bounding box for the image
	\fbox{\includegraphics[width=0.95\linewidth]{curve}}
	\caption{Figure taking up 95\% width of the column}
	\label{fig:graph-1}
\end{figure}

% Using \begin{figure*} makes the figure take up the entire width of the page
% But be careful, the figure always comes at the top of the page ... 
% so it may appear on the next available page
\begin{figure*}[hbt]\centering
	\includegraphics[width=\linewidth]{prof_ale}
	\caption{Placing a wide picture (Discouraged! as it always appears at the 
			 top of a page.)}
	\label{fig:ale-sir}
\end{figure*}

Figure~\ref{fig:graph-1} takes up 95\% of the width of a column and 
Figure~\ref{fig:ale-sir} takes the width of the entire width of the page.

%forcing a column break
%\vfill\null

\section{Lists} \label{sec:lists}

\subsection{Simple Lists}
Simple Bulleted and Numbered lists have already been presented in 
Section~\ref{sec:getting} and Section~\ref{sec:figures} respectively.

\subsection{Nested Lists}
Lists can be nested upto three levels in \LaTeX.

\subsubsection{Numbered Nested List}
Here is a nested numbered list:

\begin{enumerate}
  \item Fruits
    \begin{enumerate}
      \item Apple
      \item Orange
    \end{enumerate}
  \item Vegetables
    \begin{enumerate}
      \item Spinach
      \item Carrot
    \end{enumerate}
\end{enumerate}

\subsubsection{Bulleted Nested List}
Here is a nested bulleted list:

% [noitemsep] removes whitespace between the items for a compact look
\begin{itemize}[noitemsep] 
  \item Fruits
    \begin{itemize}[noitemsep] 
      \item Apple
      \item Orange
    \end{itemize}
  \item Vegetables
    \begin{itemize}[noitemsep] 
      \item Spinach
      \item Carrot
    \end{itemize}
\end{itemize}

\subsubsection{Mixed Nested List}
Here is a mixed nested list:
\begin{enumerate}[noitemsep]
  \item Fruits
    \begin{itemize}[noitemsep] 
      \item Apple
      \item Orange
    \end{itemize}
  \item Vegetables
    \begin{itemize}[noitemsep] 
      \item Spinach
      \item Carrot
    \end{itemize}
\end{enumerate}


\subsection{Description List}
This is for dictionary-like word and description list.
\begin{description}
  \item[Word] Definition ...
  \item[Concept] Explanation ...
  \item[Idea] Text ...
\end{description}


\section{Paragraphs with heading}

\paragraph{Hello} Place your paragraph heading inside the curly braces and your 
paragraph text here.


\section{Referencing}
The list of references should be produced using BibTeX. The BibTeX entries 
should be placed in the "refs.bib" file. Please refer BibTeX manuals or online 
resources on creating bibliography databases using BibTeX and citation. You can 
easily create bibliography database files using the GUIs like TeXMaker or JabRef. 
You can even search for BibTeX entries for a majority of publications at Google 
Scholar in the following url: 

\url{http://scholar.google.com}

Examples: This is citation one\cite{lamport1994} and these are two citations in 
one \cite{oetiker2001not, kopka1995guide}.


\section{Compilation} \label{sec:compile}
Since, this template contains citations and cross referencing along with 
reference list generated via BibTeX, the \LaTeX\ source file should be processed
four times in the following sequence to generate the final pdf output.
\begin{enumerate}
	\item PDFLatex
	\item BibTeX
	\item PDFLatex
	\item PDFLatex
\end{enumerate}

Do not worry, if there is an extra blank page at the end of the paper, this is 
an intended behavior. It happens to make the number of pages of the paper even, 
if the paper ends in an odd-numbered page. This is to make sure that every other
article always starts with an odd-numbered page.


\section{Submission} \label{sec:submit}
Before submitting the paper, the source file must be compiled without any error. 
The files that need to be submitted are:
\begin{itemize}[noitemsep]
\item article.tex (with your content)
\item article.pdf (with your content)
\item pagenum.tex (as-is)
\item paperinfo.tex (with your content)
\item refs.bib (with your content)
\item IOEGC.cls (as-is)
\item Graphics folder  (with your content)
\item Assets folder (with your content)
\end{itemize}
All these should be placed in compressed / zipped folder and submitted 
electronically.

\section{Review}
Your paper will be peer reviewed in blind by expert(s) before the conference. 
Comments may be provided in the submitted pdf file. You have to re-submit your 
paper by recompiling the \LaTeX\ source file as described in 
section~\ref{sec:compile} and submit as described in section \ref{sec:submit}.

\phantomsection
\section*{Still Having Problem?} 
\addcontentsline{toc}{section}{Problem?} 
There are a lot of online tutorials on \LaTeX\ available for free download. 
One of them being \textit{LaTeX Tutorials -- A Primer} by Indian \TeX\ Users 
Group \cite{ltxprimer}.

Further, there are websites like \url{sharelatex.com}, \url{overleaf.com}, etc.,
 which are very helpful
in finding out how to perform a specific task in \LaTeX\ .

If you still face technical problems in compiling your document in \LaTeX\ using
this template, please feel free to contact the primary author of this template 
via the following email address:

\centerline{\url{jayandra@ioe.edu.np}}

\section*{Future Enhancements}
\addcontentsline{toc}{section}{Future Enhancements} 
Lately, there has been a lot of demand for the creation of reproducible 
documents in research. One of the alternates in producing publication quality 
reproducible documents is the combination of \LaTeX\ and R-programming called 
S-weave. 

In the near future, IOEGC is planning to adapt this mechanism to support 
reproducibility of research documents. Thus, you are highly encouraged to adapt 
this philosophy starting from this edition of IOEGC.

This template has undergone a few iterations of improvement over the past few 
years and is constantly evolving. Please feel free to send in your valuable 
comments/suggestions and/or feature requests via email to the primary author 
of this template.

\phantomsection
\section*{Acknowledgments} 
% The \section*{...} command prevents section numbering, and entry in TOC
% Add the following line to show this section in TOC
\addcontentsline{toc}{section}{Acknowledgments} 
The authors are thankful to ...


\phantomsection
\bibliographystyle{unsrt}
\putbib{refs}

\vfill\null
\end{document}